\documentclass[a4paper,12pt,titlepage]{article}
\usepackage[utf8]{inputenc}
\usepackage[margin=0.725in]{geometry}
\usepackage{amsmath}
\usepackage{amssymb}
\usepackage{fancyhdr}
\usepackage{amsfonts}
\usepackage[british]{babel}
\usepackage{multirow}
\usepackage{datetime}
\usepackage{xcolor}
\usepackage[hidelinks]{hyperref}
\usepackage{url}
%%\usepackage{hyperref}
\pagestyle{fancy}
\setlength{\headheight}{15pt}

\title{User guide to invitebot}
\author{LO Kam Tao Leo\\\href{mailto:leolo@leolo.org}{leolo@leolo.org}}
\date{\today}
\begin{document}
	\maketitle
\tableofcontents\newpage
\section{Preface}
invitebot is an IRC bot developed to helps stopping spambot attacks. It helps by the main channel forwarding user to a holding channel, such user will receives an challenge via notice. If the user solves the challenge, the user will be invited to the main channel.
\subsection{Bug tracker}
The bug tracker for this project is the GitHub one, at \url{https://github.com/ktllo/inviteBot}.
\section{Downloading invitebot}
The recommended way to download invitebot is though git. The git repository is at \url{https://github.com/ktllo/inviteBot.git}. You may built invitebot via maven, by \texttt{mvn compile package}. The first build may takes a lot of time because this project has a lot of dependency.

If you cannot use maven, you can download the following library,and compile it yourself
\begin{itemize}
	\item JUnit 3.8.1
	\item guava 18.0
	\item commons-codec 1.10
	\item PircBotX 2.0.1
	\item SLF4J 1.7.10
	\item Log4j 2.2
	\item org.ivartj.args, available at \url{https://github.com/ivartj/args-java}
\end{itemize}
\section{Configuration Guide}
The configuration file has a default name ``\texttt{setting.properties}''
\subsection{Basic IRC parameters}
\paragraph{server} The address of the IRC server, it can be either a domain, or an IP address. The use of IPv6 haven't be tested.
\paragraph{port} The port number for the IRC server, the default value is 6697
\paragraph{ssl} Either \texttt{true} or \texttt{false}, to state is SSL being used for the connection. When SSL is being used, SASL-plain will be also used
\paragraph{password} The server password, or, when SSL is being used,the NickServ password
\paragraph{username} The username which will be used in SASL authentication
\paragraph{nick} The nickname for the bot, it can be changed after connection. If the stated nick is being used, a new nick, which has numbers  appended to the given nick will be used.
\paragraph{ident} The ident which will be sent to the server. The \texttt{identd} server \textbf{WILL NOT} be started. 
\end{document}